%%%% Input Codes %%%%   
% <mcode.sty> required for Matlab codes
% template for inserting codes:

% \lstset{numbers=left,numberstyle= \tiny}
% \lstinputlisting[firstline=6, lastline=15]{/Users/name/Desktop/FILENAME.M}

% or:

%\lstset{numbers=left,numberstyle= \tiny}
%\begin{lstlisting}[xleftmargin=2em,xrightmargin=2em,aboveskip=1em]
%clc; clear all; close all;
%\end{lstlisting}

\section{Basic Model Set up}

The key technical things for Heterogeneous Agent Models are distribution of agents, which relates to individual choices and aggregate variables is an {\color{red} infinite dimensional object}(many state variables). 

For every type $i$ agent we have the general formula of equilibrium: 
\[ 
E_t f(x^i_{t-1}, x^i_{t}, x^i_{t+1},u^i_{t},u^i_{t+1},x_{t-1}, x_{t}, x_{t+1},u_{t},u_{t+1})=0, 
\]


where, $x$ is endogenous variables, $u$ denotes exogenous. 

And for aggregate and individual endogenous variables, we have: 
\[
x_t \equiv \int \gamma (x_t^i)d_t (i)di \quad \text{with } \int d_t (i)di = 1 
\]

Law of motion of the density function (share of each type $i$ agent): 
\[ d_t (\cdot) = \mathcal{P}(d_{t-1}(\cdot), u_t)\]

\subsection{Krusell\&Smith Model}

Individual endogenous variables: $x^i_t = \{c^i_t,a^i_t\}$;\\

Aggregate endogenous variables: $x_t = W_t,R_t,K_t,Y_t$;\\

Exogenous variables: idiosyncratic shocks $s^i_t$ and TFP shocks $z_t$.\\

HH foc: 
\begin{align}
\begin{split}
(c^i_t)^{-\sigma} = \beta E_t R_{t+1}(c^i_{t+1})^{-\sigma} \quad \text{if } a^i_t > a_{min} \\
(c^i_t)^{-\sigma} > \beta E_t R_{t+1}(c^i_{t+1})^{-\sigma} \quad \text{if } a^i_t = a_{min} \\
c^i_t + a^i_t = s^i_t W_t + a^i_{t-1} R_{t-1}
\end{split}\label{hhfoc}
\end{align}

by Firm Maximization problem: 
\begin{align}
\begin{split}
W_t = (1-\alpha)Y_t/L_{t} \\
R_t = \alpha Y_t/K_{t-1} + (1- \delta)
\end{split}
\end{align} 


And market clearing condition for assets: $\int a^i_t d_t (i)di = K_t$\\

Above is a way of describing the models very generally. In {\color{red} PS1 and PS2}, we solve KS model by using \textbf{Krusell-Smith Algorithm with conventional VFI}, in the Literature later on, there's some extension based on Krusell-Smith Algorithm also. In this Final Project, I want to implement another which is frequently used recently in the research, the projection and perturbation methods.\\

\section{Literature Review}\label{secLiter}

One strand of literature approximates the cross-sectional distribution using a parametric family, like Algan, Allais and Den Haan (2008)\cite{algan2008solving} in which they solve for the dynamics of the distribution using a globally accurate projection technique.\\

Another strand of literature uses a mix of globally accurate and locally accurate approximations to solve for the dynamics of heterogeneous agent models like Reiter (2009)\cite{reiter2009solving}, he solves the individual problem with a projection method, and approximates the law of motion of aggregate variables and the distribution with a perturbation method.\\

However, approximates the distribution with a fine histogram, which requires many parameters to achieve acceptable accuracy. This limits Reiter (2009)\cite{reiter2009solving} approach to problems which have a low-dimensional individual state space because the size of the histogram grows exponentially in the number of individual states. More advanced things have been done by Winberry, Thomas (2018)\cite{winberry2018method}, where he approximates the distribution with a flexible parametric family, reducing its dimensionality to a finite set of endogenous parameters, and solve for the dynamics of these endogenous parameters by perturbation. This part of work can be found in  \href{http://faculty.chicagobooth.edu/thomas.winberry/research/index.html}{Winberry's website}\footnote{Thomas also provide codes for Benchmark RBC model with Firm Heterogeneity}

\section{Projection and Perturbation: Reiter Method}

Now, we only {\color{red} discretized individual state space into grids}, and also the density function such that $\sum_{i=1}^{N} d^i_t = 1$ which is like discretization the continuous distribution with "histograms". And then, we could approximate individual policy function using a projection method:

\[ 
x^i_t \approx \tilde{g}(x^i_{t-1},u^i_t, \theta_t) \quad \text{for } x^i_t \equiv g(x^i_{t-1}, u^i_t, x_{t-1},u_t,d_{t-1}), 
\]

where $\theta_t$ is the vector of coefficients. 

\subsection{The Algorithm}

\subsubsection*{Step 1: Solve for SS}

By discretization, we can solve individual policy function $x^i_t \approx \tilde{g}(x^i_{t-1},u^i_t, \theta_t)$ with a projection method\footnote{Setting aggregate shocks to zero, guess SS of capital, then we can solve for policy function coefficients $\theta$}, and use this policy function to compute the stationary distribution of agents. Then, we check market clearing condition $K=\sum_{i=1}^{N} a^i d^i$, and update $K$ until convergence

Another thing worth to notice is that during computing the evolution of distribution, we need next period asset $a^i_{next}$, previous, I use interpolation once (faster) but mostly straightly rely on the grids of assets. Here, we use approximation based on interpolation weights $W_{ij}$ since next period assets are separated among different types on the grids: 
\[
 W_{ij}=
 \begin{cases}
 1- \frac{a^i_{next} - a_j}{a_{j+1}-a_j} \quad &\text{if } a^i_{next} \in [a_j, a_{j+1}]\\ 
 \frac{a^i_{next} - a_{j-1}}{a_{j}-a_{j-1}} \quad &\text{if } a^i_{next} \in [a_{j-1}, a_{j}] \\
 0  \quad &\text{otherwise } 
 \end{cases}. 
\]

Up to now, we have girds, predetermined transition probabilities, and obtained individual policy function, we can compute the {\color{red} evolution of distribution}: 
\[ \mathcal{P}_{i'|i} = w_{ij} \times Prob(s'| s^i), \]

where distribution of agents evolves accroding to $d_t = \mathcal{P'} d_{t-1}$. 

Now, we check market clearing condition: $residuals \equiv K - \sum_{i=1}^{N}a^i d^i$

\textbf{Recap about the selection on} $\mathcal{P}_{i'|i}$: if we assume H=2 histories of agents are the same, then we have: 
\[ 
\mathcal{P}_{i'|i}=
\begin{bmatrix}
& BB & BG & GB & GG \\
BB & p_B & p_B & 0 & 0\\ 
BG & 0 & 0 & 1-p_G & 1-p_G\\
GB & 1-p_B & 1-p_B & 0 & 0\\
GG & 0 & 0 & p_G & p_G\\
\end{bmatrix}
\]

Then, Household problem in Equation \ref{hhfoc} becomes\footnote{for S state spaces power of H same-histories type of agents: $I = S^H$}: 

\begin{align}
\begin{split}
(c^i_t)^{-\sigma} = \beta R_t E_t \sum_{j=1}^I \mathcal{P}_{ji} (c^i_{t+1})^{-\sigma} \quad \text{if } a^i_t > a_{min} \\
c^i_t + a^i_t = s^i_t W_t + R_{t-1} \sum_{j=1}^I \mathcal{P}_{ji} a^j_{t-1} \frac{d^j_{t-1}}{d^i_t} \\
d^i_t = \sum_{j=1}^N \mathcal{P}_{ji} d^j_{t-1}
\end{split}
\end{align}

\subsubsection*{Step 2: compute First Order Approximation of the model}

\subsubsection*{Step 3: Solve the linearized system of equation }
Now, by "histogram" the density function, aggregate endogenous variable $x_t$ can be approximated  as $\tilde{x_t} = (\theta_t, x_t, d_t)$, and update by:
\[ \tilde{x}_t \equiv G_x \tilde{x}_{t-1} + G_u u_t\]

\newpage
\section{Quantitative Performance}


%\begin{figure}[H]
%\caption{Bond-price menu}
%\hspace{-2.0cm}
%\begin{center}
%\begin{tabular}{cc}
%\multicolumn{1}{c}{(a) } &  
%\multicolumn{1}{c}{(b) }\\
%\includegraphics[angle=0,width=.5\textwidth]{img/.png}   & 
%\includegraphics[angle=0,width=.5\textwidth]{img/.png} 
%\end{tabular}
%\end{center}
%\label{fig:4}
%\end{figure}



%\begin{figure}[H]
%\caption{}
%\hspace{-2.0cm}
%\begin{center}
%\includegraphics[angle=0,width=.67\textwidth]{img/.png}
%\end{center}
%\end{figure}

 